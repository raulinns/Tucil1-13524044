\section{Penutup}
\subsection{Lampiran}\label{sub:Lampiran} % (fold)
Tautan kode sumber yang saya rancang dalam program ini ada pada tautan berikut. \\
\noindent Link Github: \url{https://github.com/raulinns/Tucil1-13524044.git}

% subsection Lampiran (end)
\subsection{Kata Penutup}\label{sub:Kata Penutup} % (fold)
"Tugas Kecil saja sudah seperti ini, bagaimana jika Tugas Besar, ya." Itu yang aku bayangkan saat mengerjakan tugas ini. Namun, Tugas Kecil ini sangat membuka pikiranku tentang algoritma. Selama liburan, aku berpikir bagaimana cara mengurangi penggunaan AI. Melihat mata kuliah Strategi Algoritma ini, tampaknya mata kuliah ini akan menjadi gerbang menuju hal tersebut. Terima kasih, IF2211 Strategi Algoritma!

% subsection Kata Penutup (end)

\subsection{Pernyataan}\label{sub:Pernyataan} % (fold)
Tugas ini disusun sepenuhnya tanpa bantuan kecerdesan buatan (\textit{Generative AI}), melainkan hasil pemikiran dan analisis mandiri.

\raggedleft
Bandung, 18 Februari 2026

\begin{figure}[H]
    \raggedleft
    \includegraphics[width=0.13\textwidth]{figures/ttd endra.jpg}
    \hspace{1cm}
\end{figure}

\raggedleft
Narendra Dharma Wistara M. \\
13524044
% subsection Pernyataan (end)
