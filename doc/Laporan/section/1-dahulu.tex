\section{Deskripsi Penugasan}\label{sec:Pendahuluan} % (fold)
Queens adalah gim logika yang tersedia pada situs jejaring
profesional LinkedIn. Tujuan dari gim ini adalah menempatkan
\textit{queen} pada sebuah papan persegi berwarna sehingga terdapat
hanya satu \textit{queen} pada tiap baris, kolom, dan daerah warna.
Selain itu, satu \textit{queen} tidak dapat ditempatkan bersebelahan
dengan \textit{queen} lainnya, termasuk secara diagonal.

Tugas anda adalah membuat program yang dapat menemukan satu solusi
penempatan \textit{queen} pada suatu papan berwarna yang diberkan
atau menmpilkan bahwa tidak ada solusi yang valid. Program melakukan
solusi menggunakan algoritma brute force.

Misal, diberikan papan sebagai berikut. Untuk tugas ini, papan selalu
dimulai kosong.
\begin{figure}[htbp]
    \begin{center}
        \includegraphics[width=0.5\textwidth]{figures/input.png}
    \end{center}
    \caption{Papan Permainan Queens}
\end{figure}

Di bawah ini satu-satunya solusi valid. Perhatikan bahwa tiap baris,
kolom, dan daerahj warna sudah memiliki satu ditempati satu \textit{queen}.

\begin{figure}[htbp]
    \begin{center}
        \includegraphics[width=0.5\textwidth]{figures/output.png}
    \end{center}
    \caption{Solusi Permainan Queen}
\end{figure}

% section Pendahuluan (end)

\section{Spesifikasi Tugas}\label{sec:Spesifikasi Tugas} % (fold)
\subsection{Spesifikasi Wajib}\label{sub:Spesifikasi Wajib} % (fold)
\begin{itemize}[label=\textbullet]
    \item Buatlah sebuah program sederhana dalam bahasa Java, Python,
        C++, C, atau GO yang mengimplementasikan \textbf{algoritma
        \textit{Brute Force}} untuk mencari solusi dalam permainan
        \textit{Queens}.
    \item Algoritma \textit{brute force} yang dimaksud harus bersifat
        murni, tidak menggunakan heuristik apapun.
    \item \textbf{Input:} program akan memberikan pengguna sebuah
        arahan pada terminal (jika tidak mengerjakan bonus) untuk
        memilih file test case berekstensi .txt, kemudian program
        membaca file test case tersebut yang berisi konfigurasi awal
        dari papan \textit{Queens} yang akan diselesaikan. Perhatikan
        bahwa input dapat memiliki ukuran papan yang berbeda-beda,
        serta program harus dapat memvalidasi apakah input yang
        diberikan merupakan input valid atau bukan.
\end{itemize}

\noindent Berikut adalah contoh file .txt yang akan dijadikan sebagai input:

\begin{tcolorbox}[colback=white, colframe=black, sharp corners, boxrule=0.5pt]
\begin{verbatim}
AAABBCCCD
ABBBBCECD
ABBBDCECD
AAABDCCCD
BBBBDDDDD
FGGGDDHDD
FGIGDDHDD
FGIGDDHDD
FGGGDDHHH
\end{verbatim}
\end{tcolorbox}

\begin{itemize}[label=\textbullet]
    \item \textbf{Output:}
        \begin{enumerate}
            \item Menampilkan hasil akhir dari papan yang sudah
                terisi oleh \textit{queens} dengan aturan yang benar.
            \item Menampilkan banyak konfigurasi atau iterasi yang
                ditinjau oleh algoritma.
            \item Menampilkan waktu eksekusi program dalam
                \textit{millisecond} (cukup waktu pencarian dengan
                algoritma, tidak termasuk membaca dan menulis file).
            \item \textit{Live Update:} program harus dapat
                memvisualisasikan proses \textit{brute force} yang dilakukan.
        \end{enumerate}
\end{itemize}

\noindent Berikut adalah contoh output berdasarkan contoh input diatas:

\begin{tcolorbox}[colback=white, colframe=black, sharp corners, boxrule=0.5pt]
\begin{verbatim}
AAABBCC#D
ABBB#CECD
ABBBDC#CD
A#ABDCCCD
BBBBD#DDD
FGG#DDHDD
#GIGDDHDD
FG#GDDHDD
FGGGDDHH#

Waktu pencarian: 120 ms
Banyak kasus yang ditinjau: 1000 kasus
Apakah Anda ingin menyimpan solusi? (Ya/Tidak)
\end{verbatim}
\end{tcolorbox}

% subsection Spesifikasi Wajib (end)

\subsection{Spesifikasi Bonus}\label{sub:Spesifikasi Bonus} % (fold)
Poin maksimal untuk bonus adalah 10.
\begin{itemize}
    \item Membuat GUI (\textit{Graphical User Interface} untuk
            program secara keseluruhan yang dapat memberi
            \textit{input} dan memberi visualisasi (8 poin).
        \item \textit{Input as image}, \textit{input} dari program
            adalah sebuah gambar yang merupakan konfigurasi dari
            papan awal. (5 poin)
        \item \textit{Output as image}, \textit{output} dari hasil
            peletakan \textit{queens} diberikan dalam bentuk gambar. (2 poin)
    \end{itemize}

    % subsection Spesifikasi Bonus (end)
    % section Spesifikasi Tugas (end)

    \section{Dasar Teori}\label{sec:Dasar Teori} % (fold)
    \subsection{Brute Force}\label{sub:Brute Force} % (fold)
    Algoritma \textit{Brute Force} adalah sebuah pendekatan yang
    menyelesaikan persoalan secara lempang
    (\textit{straightforward}). Algoritma ini didasarkan langsung
    pada pernyataan persoalan (\textit{problem statement}) dan
    definisi konsep yang terlibat. Secara harfiah, kata
    "\textit{force}" mengindikasikan penggunaan "tenaga" komputasi
    ketimbang kecerdasan algoritma. Algoritma ini lebih cocok untuk
    persoalan dengan ukuran masukan yang kecil, karena
    langkah-langkah penyelesaiannya cenderung sederhana dan
    implementasinya mudah. Algoritma \textit{brute force} dapat
    diterapkan untuk memecahkan semua persoalan dan menjamin solusi
    persoalannya, jika ada.
    % subsection Brute Force (end)

    \subsection{Multithreading dalam Java}\label{sub:Multithreading} % (fold)
    \textit{Multithreading} adalah kemampuan sebuah program untuk
    menjalankan beberapa tugas secara paralel dalam satu waktu. Dalam
    JavaFX, terdapat satu \textit{thread} utama yang disebut JavaFX
    Application Thread yang bertanggung jawab untuk menangani UI.
    \textit{Multithreading} digunakan agar algoritma \textit{brute
    force} tetap berjalan selagi tampilan di GUI tetap terbarui
    (\textit{Live Update}). Dalam Java, \textit{multithreading} dapat
    diimplementasikan dengan membuat objek \texttt{Thread} yang sebuah fungsi.

\begin{lstlisting}[language=Java, basicstyle=\ttfamily\small, breaklines=true]
    new Thread(() -> {
        // Pemanggilan fungsi untuk Thread utama
        Solver.solve() -> {
            // Kembali ke fungsi lain
        Platform.runLater(() -> Main.GUI());
    });
}).start();
\end{lstlisting}

    % subsection Multithreading (end)
    % section Dasar Teori (end)
