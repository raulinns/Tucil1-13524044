\section{Desain}\label{sec:Desain} % (fold)
\subsection{Tech Stack}\label{sub:Tech Stack} % (fold)
Aplikasi ini dirancang menggunakan bahasa Java untuk memungkinkan program dapat berjalan di berbagai platform, serta mempermudah perancangan GUI (spesifikasi bonus). Secara garis besar, arsitektur aplikasi terdiri dari komponen-komponen berikut.
\begin{enumerate}
    \item Bahasa Pemrograman: Java 17
    \item Framework GUI: JavaFX
    \item Menejemen Dependensi: Apache Maven 
\end{enumerate}
% subsection Tech Stack (end)

\subsection{Struktur Program}\label{sub:Struktur Program} % (fold)
Proyek disusun mengikuti standar Maven dan spesifikasi tugas:
\begin{enumerate}
    \item \texttt{src/main/}
        \begin{enumerate}
            \item \texttt{src/main/java/queens/}: Kode sumber Java.
            \item \texttt{src/main/input/}: \textit{Test case} yang digunakan untuk melakukan \textit{testing}.
        \end{enumerate}
    \item \texttt{bin/}: \textit{File executable} atau hasil kompilasi.
    \item \texttt{doc/}: Spesifikasi dan laporan tugas kecil dalam PDF
    \item \texttt{pom.xml}: Konfigurasi Maven dan daftar dependensi \textit{library}.
    \item \texttt{doc/}: Dokumentasi laporan PDF.
    \item \texttt{pom.xml}: Konfigurasi Maven dan daftar dependensi \textit{library}.
\end{enumerate}
% subsection Struktur Program (end)
Program utama dibagi menjadi dua komponen utama untuk memisahkan algoritma dengan GUI:
\begin{itemize}
    \item \texttt{queens.Solver} berisi algoritma \textit{brute force} untuk mmenempatkan \textit{queens}. Terdapat logika \textit{solver}
    \item \texttt{queens.Main} berisi komponen UI yang dibangun dengan JavaFX. 
\end{itemize}

\subsection{Cara Kompilasi dan Menjalankan Program}\label{sec:Cara Kompilasi dan Menjalankan Program} % (fold)
Untuk mengompilasi seluruh kode sumber dan mengunduh dependensi JavaFX, jalankan perintah berikut pada direktori \textit{root} projek.
\begin{lstlisting}[language=bash, basicstyle=\ttfamily\small, breaklines=true]
    mvn clean compile
\end{lstlisting}

\noindent Untuk menjalankan GUI, gunakan JavaFX dengan perintah berikut. \begin{lstlisting}[language=bash, basicstyle=\ttfamily\small, breaklines=true]
    mvn javafx:run
\end{lstlisting}

\noindent Untuk menjalankan aplikasi melalui CLI, gunakan perintah berikut.
\begin{lstlisting}[language=bash, basicstyle=\ttfamily\small, breaklines=true]
    java -cp target/classes queens.Solver
\end{lstlisting}

% subsection Cara Kompilasi dan Menjalankan Program (end)
% section Desain (end)
