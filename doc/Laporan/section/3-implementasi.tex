\section{Implementasi Spesifikasi Wajib}\label{sec:Implementasi Spesifikasi Wajib} % (fold)
\subsection{Algoritma Utama}\label{sub:Algoritma Utama} % (fold)

Algoritma UtamaAlgoritma ini akan mencoba untuk menempatkan satu ratu pada setiap baris secara sistematis dan memeriksa validitasnya. Untuk membantu algoritma ini, dibuat beberapa struktur data, yaitu:
\begin{itemize}
    \item \texttt{Board} digunakan sebagai representasi fisik dari papan permainan. Struktur data ini diimplementasikan menggunakan matriks atau \textit{array} dua dimensi. Setiap elemen dalam \textit{array} menyimpan warna dalam petak.
\begin{lstlisting}[language=Java, basicstyle=\ttfamily\small, breaklines=true]
    public static String[][] board;
\end{lstlisting}

    \item \texttt{Koordinat} digunakan untuk menyimpan koordinat pada \textit{board}. \texttt{Koordinat} diimplementasikan menggunakan struktur data \textit{record} dari Java. Setiap \textit{record} menyimpan koordinat dalam papan dengan $x$ sebagai baris dan $y$ sebagai kolom. 
\begin{lstlisting}[language=Java, basicstyle=\ttfamily\small, breaklines=true]
    public record Koordinat(int x, int y) {}
\end{lstlisting}
\item \texttt{Koordinat[] queenLocation} digunakan untuk menyimpan \textit{set of queens} yang sedang divalidasi.
    \begin{lstlisting}[language=Java, basicstyle=\ttfamily\small, breaklines=true]
    Koordinat[] queenLocation = new Koordinat[n];
    for (int i = 0; i < queenLocation.length; i++) {
        queenLocation[i] = new Koordinat(i, 0);
    }
\end{lstlisting}

\item \textit{HashSet} \texttt{uniqueColor} yang digunakan untuk menyimpan setiap warna yang ada di papan untuk dihitung jumlah warna yang terdapat di dalam papan.
    \begin{lstlisting}[language=Java, basicstyle=\ttfamily\small, breaklines=true]
    Set<String> uniqueColor = new HashSet<>();
\end{lstlisting}

\end{itemize}

Berikut adalah langkah-langkah algoritmanya:
\begin{enumerate}
    \item Program membuat \textit{array of Record} \texttt{queenLocation} berukuran $N$. Di mana setiap elemen array menyimpan koordinat $(x, y)$, di mana $x$ adalah baris $[0..N -1]$ dan $y$ adalah kolom yang awalnya diinisialisasi dengan $0$.
    \item Program memasuki \textit{infinite loop} dengan \texttt{while (true)} untuk mengevaluasi setiap kemungkinan penempatan ratu. Pada setiap iterasi, program mengecek apakah konfigurasi ratu saat ini sudah memenuhi aturan-aturan berikut.
        \begin{itemize}
            \item Tidak ada dua ratu di kolom yang sama
            \item Tidak ada dua ratu di wilayah warna yang sama
            \item Tidak ada dua ratu yang saling bersebelahan, termasuk secara diagonal
        \end{itemize}
    \item Jika konfigurasi memenuhi aturan, maka algoritma berhenti dan mengembalikan lokasi ratu.
    \item Jika konfigurasi tidak memenuhi aturan, maka program akan menggeser ratu
        \begin{itemize}
            \item Mulai dari baris pertama. Jika kolom belum mencapai indeks maksimal $N-1$, geser kolom ratu ke bawah sebesar 1 petak.
            \item Jika kolom sudah mencapai indeks maksimal, kembalikan kolom ke 0, lalu pindah ke baris berikutnya.
            \item Ulangi hingga seluruh kemungkinan teriterasi atau solusi ditemukan.
        \end{itemize}
    \item Jika tidak ditemukan solusi hingga akhir, maka program berhenti dan menyatakan tidak ada soluti.
\end{enumerate}

Algoritma ini memiliki kompleksitas waktu algoritma sebesar $O(N^N$, sehingga untuk $N$ yang sangat besar memakan waktu yang sangat lama.
% subsection Algoritma Utama (end)

\subsection{Algoritma Utama dalam Java}\label{sub:Algoritma Utama dalam Java} % (fold)

Algoritma Utama dalam Java\begin{lstlisting}[language=Java, basicstyle=\ttfamily\small, breaklines=true]
public static boolean solve(Koordinat[] queenLocation, Consumer<Koordinat[]> onStep) {
    int n = board.length;

    while (true) {
        totalKasus++; 
        
        if (validSolution(queenLocation)) {
            return true;
        }

        int baris = 0;
        while (baris < n) {
            int kolom = queenLocation[baris].y;
            if (kolom < n - 1) {
                queenLocation[baris] = new Koordinat(baris, kolom + 1);
                break;
            } else {
                queenLocation[baris] = new Koordinat(baris, 0);
                baris++;
            }
        }

        if (baris == n) {
            break;
        }
    }
    return false;
}
\end{lstlisting}
% subsection Algoritma Utama dalam Java (end)

\subsection{Fungsi Pembantu}\label{sub:Fungsi Pembantu} % (fold)
Untuk membantu berjalannya program utama, digunakan beberapa fungsi pembantu seperti.
\begin{enumerate}
    \item \texttt{validSolution()}. Fungsi ini mengambil input berupa \textit{list of Record} \texttt{Koordinat[] queenLocation} dan mengembalikan \texttt{boolean}. Fungsi ini berfungsi untuk mengecek apakah konfigurasi \textit{queens} saat ini sudah memenuhi syarat. 
        \begin{lstlisting}[language=Java, basicstyle=\ttfamily\small, breaklines=true]
    public static boolean validSolution(Koordinat[] queenLocation) { ... }
\end{lstlisting}

\item \texttt{checkColor()}. Fungsi ini mengambil sebuah \textit{record} dan mengembalikan \texttt{string} yang berisi label warna dari sebuah koordinat. 
    \begin{lstlisting}[language=Java, basicstyle=\ttfamily\small, breaklines=true]
    public static String checkColor(Koordinat queen) { ... }
\end{lstlisting}

\item \texttt{colorCount()}. Fungsi ini mengembalikan \texttt{integer} jumlah total warna yang ada dalam papan.
\begin{lstlisting}[language=Java, basicstyle=\ttfamily\small, breaklines=true]
    public static int colorCount() {
        Set<String> uniqueColor = new HashSet<>();
        ...
        return uniqueColor.size(); 
    }
\end{lstlisting}

\item \texttt{loadBoard()}. Fungsi ini mengambil sebuah \texttt{string} berisi \textit{path file} \texttt{.txt} dan mengembalikan \texttt{boolean} apakah \textit{file} tersebut merupakan konfigurasi papan yang valid.
    \begin{lstlisting}[language=Java, basicstyle=\ttfamily\small, breaklines=true]
    public static boolean loadBoard(String filePath) {
        ...
        return board.length == colorCount(); 
    }
\end{lstlisting}

\item \texttt{printBoard()}. Fungsi ini mengambil sebuah konfigurasi ratu dan mencetak papan ke terminal.
\begin{lstlisting}[language=Java, basicstyle=\ttfamily\small, breaklines=true]
    public static void printBoard(Koordinat[] queenLocation) {
       if (isQueen) System.out.print("#");
        else System.out.print(board[i][j]);
    }
\end{lstlisting}
\end{enumerate}
% subsection Fungsi Pembantu (end)
% section Implementasi Spesifikasi Wajib (end)

