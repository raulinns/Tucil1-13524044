\section{Implementasi Spesifikasi Bonus}\label{sub:Implementasi Spesifikasi Bonus} % (fold)
\subsection{GUI}\label{sub:GUI} % (fold)

Dalam \texttt{Main.java}, struktur komponen digunakan sebagai berikut.
\begin{itemize}
    \item Stage \& Scene sebagai kontainer utama.
    \item VBox (Root). Layout utama untuk elemen disusun secara vertikal.
    \item Hbox. Baris horizontal berisi \texttt{TextField} untuk \textit{input} nama \textit{file} dan \texttt{Button} untuk memulai penyelesaian.
    \item GridPane. Komponen untuk me-\textit{render} papan.
\end{itemize}

Untuk memastikan GUI tetap responsif saat menjalankan algoritma, program ini menerapkan \textit{multithreading} untuk memisahkan perhitungan algoritma dengan menampilkan hasil perhitungan.
\begin{enumerate}
    \item Algoritma \texttt{Solver.solve} dijalankan di dalam \textit{thread} terpisah.
    \item \texttt{Solver} menerima objek \texttt{Consumer}.
    \item \texttt{Platform.runLater()} memperbarui tampilan papan dari \textit{thread} dari nomor 1.
\end{enumerate}

Fungsi \texttt{renderBoard} merupakan fungsi utama yang menampilkan \texttt{String[]][] board} dari \texttt{Solver} dalam GUI.
\begin{lstlisting}[language=Java, basicstyle=\ttfamily\small, breaklines=true]
     private void renderBoard(String[][] board, Solver.Koordinat[] solution) {
        boardGrid.getChildren().clear();
        int n = board.length;

        double cellSize = Math.min(600.0 / n, 50.0);
        double fontSize = cellSize * 0.5;

        generateColorMap(board);

        for (int i = 0; i < n; i++) {
            for (int j = 0; j < n; j++) {
                StackPane cell = new StackPane();
                Rectangle rect = new Rectangle(cellSize, cellSize);
                rect.setFill(colorMap.getOrDefault(board[i][j], Color.LIGHTGRAY));
                rect.setStroke(Color.BLACK);
                rect.setStrokeWidth(0.2); 
                cell.getChildren().add(rect);

                for (Solver.Koordinat q : solution) {
                    if (q.x() == i && q.y() == j) {
                        Label queenLabel = new Label("#");
                        queenLabel.setStyle(
                                "-fx-font-size: " + fontSize + "px; -fx-text-fill: black;");
                        cell.getChildren().add(queenLabel);
                    }
                }
                boardGrid.add(cell, j, i);
            }
        }
    }
\end{lstlisting}

Algoritma ini merubah koordinat dalam \texttt{board} dengan \textit{cell} dengan warna yang berbeda-beda yang diberikan oleh fungsi \texttt{generateColorMap}.
% subsection GUI (end)

\subsection{Output as Image}\label{sub:Output as Image} % (fold)
\begin{lstlisting}[language=Java, basicstyle=\ttfamily\small, breaklines=true]
    private void saveAsImage(String fileName) {
    try {
        WritableImage snapshot = boardGrid.snapshot(null, null);
        File outputFile = new File("src/main/output/" + fileName + ".png");
        ImageIO.write(SwingFXUtils.fromFXImage(snapshot, null), "png", outputFile);
        System.out.println("Papan berhasil disimpan sebagai gambar: " + outputFile.getAbsolutePath());
    } catch (Exception e) {
        System.out.println("Gagal menyimpan gambar: " + e.getMessage());
    }
}
\end{lstlisting}

Algoritma ini mengambil tangkapan dari \texttt{boardGrid}, lalu menyimpannya dalam \textit{folder} \texttt{src/main/output/}. Untuk keseragaman \textit{output} dikonversi menuju format \texttt{.png}.
% subsection Output as Image (end)
% section Implementasi Spesifikasi Bonus (end)
