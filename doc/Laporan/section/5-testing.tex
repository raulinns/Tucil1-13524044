\section{Pengujian}\label{sec:Pengujian} % (fold)
Untuk memastikan program berjalan dengan benar, dilakukan beberapa
pengujian dengan beberapa kasus uji (\textit{test case}. Kasus uji
    yang digunakan adalah sebagai berikut.
    \begin{enumerate}
        \item Kasus Uji $4 \times 4$
        \item Kasus Uji $9 \times 9$
        \item Kasus Uji $10 \times 10$
        \item Kasus Uji Tidak Valid
            \begin{itemize}
                \item Papan Sangat Sempit
                \item Jumlah Baris $\ne$ Kolom
                \item Jumlah Baris $\ne$ Jumlah Warna
                \item \textit{File input} tidak valid
            \end{itemize}
    \end{enumerate}

    \pagebreak

    \subsection{Kasus Uji 1}\label{sub:Kasus Uji 1} % (fold)

    \begin{tcolorbox}[colback=white, colframe=black, sharp corners,
        boxrule=0.5pt]
\begin{verbatim}
AABB
AABB
CCDD
CCDD
\end{verbatim}
    \end{tcolorbox}

    \begin{itemize}
        \item Deskripsi: Papan $4 \times 4$
        \item Output:
            \begin{figure}[htbp]
                \begin{center}
                    \includegraphics[width=\textwidth]{figures/tc1.png}
                \end{center}
                \caption{Hasil Kasus Uji 1}
            \end{figure}

    \end{itemize}
    % subsection Kasus Uji 1 (end)

    \subsection{Kasus Uji 2}\label{sub:Kasus Uji 2} % (fold)
    \begin{tcolorbox}[colback=white, colframe=black, sharp corners,
        boxrule=0.5pt]
\begin{verbatim}
AAABBCCCD
ABBBBCECD
ABBBDCECD
AAABDCCCD
BBBBDDDDD
FGGGDDHDD
FGIGDDHDD
FGIGDDHDD
FGGGDDHHH
\end{verbatim}
    \end{tcolorbox}

    \begin{itemize}
        \item Deskripsi: Papan $9 \times 9$
        \item Output:
            \begin{figure}[htbp]
                \begin{center}
                    \includegraphics[width=0.7\textwidth]{figures/tc2.png}
                \end{center}
                \caption{Hasil Kasus Uji 2}
            \end{figure}

    \end{itemize}

    % subsection Kasus Uji 2 (end)

    \subsection{Kasus Uji 3}\label{sub:Kasus Uji 3} % (fold)
    \begin{tcolorbox}[colback=white, colframe=black, sharp corners,
        boxrule=0.5pt]
\begin{verbatim}
ABCDEFGHIJ
ABCDEFGHIJ
ABCDEFGHIJ
ABCDEFGHIJ
ABCDEFGHIJ
ABCDEFGHIJ
ABCDEFGHIJ
ABCDEFGHIJ
ABCDEFGHIJ
ABCDEFGHIJ
\end{verbatim}
    \end{tcolorbox}

    \begin{itemize}
        \item Deskripsi: Papan $10 \times 10$
        \item Output:
            \begin{figure}[htbp]
                \begin{center}
                    \includegraphics[width=0.7\textwidth]{figures/tc3.png}
                \end{center}
                \caption{Hasil Kasus Uji 3}
            \end{figure}

    \end{itemize}

    % subsection Kasus Uji 3 (end)

    \subsection{Kasus Uji 4}\label{sub:Kasus Uji 4} % (fold)
    \begin{tcolorbox}[colback=white, colframe=black, sharp corners,
        boxrule=0.5pt]
\begin{verbatim}
ABC
ABC
ABC
\end{verbatim}
    \end{tcolorbox}

    \begin{itemize}
        \item Deskripsi: Papan $3 \times 3$ (Papan terlalu sempit
            tidak ada solusi)
        \item Output:
            \begin{figure}[htbp]
                \begin{center}
                    \includegraphics[width=0.5\textwidth]{figures/tc4.png}
                \end{center}
                \caption{Hasil Kasus Uji 4}
            \end{figure}
    \end{itemize}
    % subsection Kasus Uji 4 (end)

    \subsection{Kasus Uji 5}\label{sub:Kasus Uji 5} % (fold)
    \begin{tcolorbox}[colback=white, colframe=black, sharp corners,
        boxrule=0.5pt]
\begin{verbatim}
AA
BB
BB
\end{verbatim}
    \end{tcolorbox}

    \begin{itemize}
        \item Deskripsi: Papan $3 \times 2$ (Jumlah Baris dan Kolom tidak sama)
        \item Output:
            \begin{figure}[htbp]
                \begin{center}
                    \includegraphics[width=0.5\textwidth]{figures/tc5.png}
                \end{center}
                \caption{Hasil Kasus Uji 5}
            \end{figure}
    \end{itemize}
    % subsection Kasus Uji 5 (end)

    \subsection{Kasus Uji 6}\label{sub:Kasus Uji 6} % (fold)
    \begin{tcolorbox}[colback=white, colframe=black, sharp corners,
        boxrule=0.5pt]
\begin{verbatim}
ABCDD
ABCDD
ABCDD
ABCDD
ABCDD
\end{verbatim}
    \end{tcolorbox}

    \begin{itemize}
        \item Deskripsi: Jumlah Baris ($5$) dengan Jumlah Warna ($4$) tidak sama
        \item Output:
            \begin{figure}[htbp]
                \begin{center}
                    \includegraphics[width=0.5\textwidth]{figures/tc6.png}
                \end{center}
                \caption{Hasil Kasus Uji 7}
            \end{figure}
    \end{itemize}

    % subsection Kasus Uji 6 (end)

    \subsection{Kasus Uji 7}\label{sub:Kasus Uji 7} % (fold)
    \begin{tcolorbox}[colback=white, colframe=black, sharp corners,
        boxrule=0.5pt]
\begin{verbatim}
    random.txt
\end{verbatim}
    \end{tcolorbox}

    \begin{itemize}
    \item Deskripsi: \textit{File input} tidak valid)
    \item Output:
        \begin{figure}[htbp]
            \begin{center}
                \includegraphics[width=0.5\textwidth]{figures/tc7.png}
            \end{center}
            \caption{Hasil Kasus Uji 7}
        \end{figure}
\end{itemize}
% subsection Kasus Uji 7 (end)

\pagebreak

\subsection{Hasil Pengujian}\label{sub:Hasil Pengujian} % (fold)
Berikut adalah ringkasan hasil pengujian kasus uji yang sudah
dilakukan beserta waktu penyelesaian dan banyak kasus yang ditinjau.
\begin{table}[!htbp]
    \begin{center}
        \begin{tabular}[\textwidth]{|c|L{4cm}|c|c|c|}
            \hline
            \textbf{No} & \textbf{Deskripsi Kasus} & \textbf{Ukuran} &
            \textbf{Waktu} & \textbf{Banyak Kasus} \\
            \hline
            1 & Kasus Normal & $4 \times 4$ & $\sim0$ ms & 115 kasus \\
            \hline
            2 & Kasus Normal & $9 \times 9$ & $304$ ms & 11.775.509 kasus \\
            \hline
            3 & Kasus Normal & $10 \times 10$ & $5159$ ms & 241.357.969 kasus \\
            \hline
            4 & Papan Terlalu Sempit & $3 \times 3$ & N/A & Tidak ada solusi \\
            \hline
            5 & Baris $\neq$ Kolom & $3 \times 2$ & N/A & Input tidak valid \\
            \hline
            6 & Baris $\neq$ Warna & $5 \times 5$ & N/A & Input tidak valid \\
            \hline
            7 & File Tidak Ditemukan & - & N/A & Input tidak valid \\
            \hline
        \end{tabular}
        \caption{Ringkasan Hasil Pengujian Kasus Uji}
    \end{center}
\end{table}
% subsection Hasil Pengujian (end)
% section Pengujian (end)
